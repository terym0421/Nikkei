\documentclass{ltjsarticle}

\title{日経新聞朝刊}
\author{}
\date{\today}

\begin{document}
\maketitle

\section{米中報復関税消耗戦に}

中国政府は米国製品への報復関税を84\%から125\%へ引き上げる.またWTOに再提訴した.

\section{京都新興企業,iPS細胞でがん治療法試験}

京都大学発の新興企業シノビ・セラぴゅーティクスはiPS細胞から免疫細胞をつくり肺や肝臓のがん細胞を攻撃する治療法の臨床試験を2026年度末に国内で開始する.

\section{ウェルシアとツルハ経営統合}

ウェルシアHDとツルハHDは12月に経営統合すると発表,当初の目標は27年度末としていたが,2年早めることとなった.

\subsection{背景}

国内市場の緊縮(人口減少,ネット市場)によって競争が激化していることから,規模を拡大してアジア市場に進出する狙い.

\subsection{詳細}

ツルハHDが株式を5分割したうえで,ウェルシアHDと\emph{株式交換}の方式で完全子会社化する予定.効力発行日は12月1日で,ウェルシアHD1株に対してツルハHDを1.15株割り当てる.(4/11終値でウェルシアHDは2,480円,ツルハHDは10,900.株価の調整によってツルハHD株は$\frac{10,900}{5}\times 1.15 = 2,507$)その後TOBを行い連結子会社化する.

\subsection{用語}

\begin{itemize}
  \item 株式交換方式とは,
\end{itemize}


\section{渋滞削減,地域再生}

国交省が交通データや地域の声などをもとに特定した「主要渋滞箇所」は2024年度で全国に8110箇所あり,12年度から1000箇所減少した.

最も減少率の高かった和歌山県では,県と整備する都市計画道路の交通分散効果が大きかった.市内の住宅地価向上へつながる


\section{EUは関税交渉と安保協力を天秤にかける状況に}

ロシアに国境の近い国々は第2のウクライナになりかねないとの不安があるなか,トランプ政権は欧州防衛への関与に消極的な姿勢である.その背景にはNATO加盟国全体の防衛費の7割をアメリカが負担しているという現状がある.

EUとしてはアメリカを対ロシアの盾として引き付けておきたい,貿易交渉で妥協したとしても,アメリカの欧州関与を維持・強化できるのなら採算がとれる,と見られている.一方,第2次大戦の反省から,EUには自由貿易を守るべきとの考えが浸透している.交渉が決裂し,EUも関税戦争に加わる可能性は否定できない.フランスは追加関税にとどまらない強硬措置(「\emph{反威圧措置(ACI)}」)を採るべきとの声がある.

別の視点から貿易摩擦から逃れようという動きもある.米中に頼りすぎない貿易関係を模索しているなか,その候補がインドである.


\section{用語}

\begin{itemize}
  \item 反威圧措置(Anti-Coercion Instrument)とは,2023年に可決された,EU加盟国に対する経済的威圧に対応する法的枠組みを提供する規則である.その具体的目的は,外国からの経済的威圧を「抑止」するものであるが,威圧が存在すると認定された場合に,EU加盟国が集団としてそれを停止させれ各種措置を発動すること,および当該国への損害賠償を請求することを可能にする.
\end{itemize}


\end{document}