\documentclass{ltjsarticle}

\title{日経新聞朝刊}
\author{}
\date{\today}

\begin{document}
\maketitle

\section{株安連鎖,日米欧500兆円超え減少}

トランプ関税の影響を受け,4日,日経平均は34,000円を割り,日米欧全体で時価総額500兆円を超える額が吹き飛んだ.

JPモルガン「1968年の\emph{歳入・支出管理法}以来の米国最大の増税となる」
NRI「OECDモデルで影響を測定したところ世界の実質gdpは3年間で0.64\%縮小させる」


\subsection{カナダに続き中国も報復関税}

中国政府はアメリカからのすべての輸入品に34\%の追加関税を設けることを発表した.

\subsection{詳細}

中国は2月と3月にすでに10~15\%の報復関税を設定指定が,それに上乗せする形となる.また中国商務省は米国の相互関税について\emph{世界貿易機関(WTO)}に提訴した.さらに電気自動車に用いられる\emph{ジスプロシウム}などの7種類のレアアースの米国に対する輸出制限をかけることとし,中国当局の許可がない限り輸出できない.

\subsection{用語}

\begin{itemize}
  \item WTOは1947年に制定されたGATT(関税および貿易に関する一般協定)を前身とする国際機関で,ウルグアイ・ラウンド交渉において設立された.
  WTOルールとはいくつかあるが,GATTの3原則(最恵国待遇・内国民待遇・貿易制限の禁止)が重要.

  \item ジスプロシウムとは,原子番号66番の原子でレアアースの一種.レーザーや商用照明に使用されている他,中性子吸収率が高いため,原子炉の制御棒に用いられている.また磁化しやすいため,ハードディスクなどに用いられている.近年では電気自動車のモーターへの活用需要が高まっている.ネオジム磁石の保磁力を高めるために,ネオジムの6\%をジスプロシウムに代替することが提案されているが,これは電気自動車1台あたり最大100gとなり,トヨタが年間200万台の生産目標を立てていることを考えると,ジスプロシウム資源はすぐに枯渇してしまう.
\end{itemize}



\section{ガソリン価格低下}

自公国民民主3党は,6月から2026年3月までガソリン価格を低額で引き下げることで合意した.

\subsection{詳細}

具体的な引き下げ額については今後調整するとしたが,現在の\emph{ガソリン補助額}を拡充する可能性がある.財源となる政府の基金は1.2兆円を見込むとし,補正予算を組まなくてよい形でどれだけ引き下げられるか検討する,とした.\emph{ガソリン税の旧暫定税率}廃止には慎重な姿勢を見せ,税制の改正には時間がかかるため,よりスピーディーにできる施策を行っていくとした.


\begin{itemize}
  \item ガソリン補助額(燃料油価格激変緩和補助金)とは,コロナ渦からの経済回復のために実施された施策であり,石油メジャーなどの石油元売りに対して補助金を与えることで価格抑制を図る.リッター170円を超えると1リットル当たり5円を上限として元売りに支給する.
  \item ガソリン税の暫定税率とは,もともとは1974年度の暫定税率として道路整備の財源に充てるための税率であった.現在では一般財源に充てられている状況となっている.1リットル当たり160円を超える場合停止される「トリガー条項」が設定されているが,3.11の復興を理由に停止されている.現在はリッター53.8円上乗せされている.
  \item 
\end{itemize}



\section{日産,アメリカで増産に一転}

日産は4日,予定していた減産計画を一部撤回することを発表した.トランプ関税への対応とみられている.

\subsection{背景}

日産は業績不振を受け,世界で構造対策を進めている.アメリカの2か所の向上では,生産ラインのシフトを半分に減らす予定だったが,それを維持することとした.日産は主力車のスポーツカー「ローグ」をアメリカで生産している.また,メキシコで生産している米国向け車種の一部受注を停止するとした.



\section{チャイナプラスワン戦略に転機}

カシオ計算機など,中国周辺領域に生産拠点を分散する「チャイナプラスワン戦略」を採用している日本製造業は,トランプ関税によって転機が訪れることとなった.


\subsection{詳細}

拠点を分散することによってサプライチェーンを安定させる戦略は自由貿易を前提としていたが,トランプ関税の発動によってそれが崩れることとなった.


\end{document}