\documentclass{ltjsarticle}

\title{日経新聞朝刊}
\author{}
\date{\today}

\begin{document}
\maketitle

\section{日産,アメリカに生産の一部を移管}

日産はトランプ関税の影響を抑えるため,米国向け主力車の国内生産を一部現地生産に切り替える検討に入った.

\subsection{感じたこと}

自動車部品メーカーなどの関連事業者にかなり影響が出る?日本の産業の空洞化が進んでしまう.アメリカの狙い通りの展開



\section{トランプ,FRBに利下げ要求}

高関税政策で景気後退への懸念が強まる中,トランプはFRBに「利下げをするなら今が絶好のタイミング」と主張している.実際に物価上昇率はまだ2\%を上回る水準である.


\subsection{感じたこと}

米国内は今後輸入物価が高騰していくため,利下げに慎重になるのは当然?一方で利下げによる物価高騰よりも高関税による景気後退の影響の方が大きいという見方もあり,その結果利下げへの期待が高まっている.


\section{コメ政策,内外から圧力}

日本の閉鎖的なコメ政策に内外から圧力が強まっている.「\emph{令和の米騒動}」を受け国内飲食業からは輸入の声が高まり,米国の日本のミニマムアクセス米への高関税を批判している.

\subsection{詳細}

1月から2月にかけての米国産コメの輸入量は7万423トンで前年比19\%増加していた.輸入米の割安さを理由に,小売や外食で外国産のコメを使う動きが広がっている.またイオンは米国産と国産米をブレンドした新商品「二穂の匠」を発売する.農水省は1月,備蓄米の放出を決めたが,価格抑制は限定的で,3月時点の米5キロの平均価格は4197円で,前年比2倍を超える.

米国産コメの主産地はカリフォルニア州で,民主党の地盤が多い.トランプ関税の対抗措置として米国産農産物に報復関税をかける国が増えたことによって,米国内農業者の不満を抑える必要が出てきた.これによって以前から閉鎖的なコメ政策をとっていた日本を標的にして支持者獲得を狙っているとみられている.


\subsection{用語}

\begin{itemize}
  \item 令和の米騒動とは,2024年夏にスーパーなどの米の棚卸が減少したことを指す.
  \item 輸入差益,日本政府は国際取り決めによって最低限米を無関税で輸入する義務を負っているが,国内の卸業者が落札時に国に事実上の関税として輸入差益を支払っている.
\end{itemize}



\section{停滞する韓国経済}

ユン大統領をめぐる問題で韓国経済は停滞を強いられた.ユン氏は大統領就任以来,日米との関係改善など外交・安全保障で手腕を破棄してきた.一方,国内経済分野では効果的な手を打てなかった.元来低かった内需に加えて,戒厳令ショックによって一層の経済停滞となった.

また少子化対策も問題であり,合計特殊出生率は0.75と世界最低水準である.

年金も改革できず,積立金の枯渇が懸念されている.

不動産価格が高騰していることも問題となっている.ソウル市内のアパート戸建ての単位面積当たりの平均販売価格は,大統領就任時から今年3月時点までに7\%上昇した.


\section{能動的サイバー防御法案}

現在国会で「能動的サイバー防御」に向けた関連法案が審議されている.「通信情報の利用」「アクセス・無害化措置」「官民連携」の3つが重要な点となる.一方で,憲法の保障する「通信の秘密」との整合性が争点となる.

「通信情報の利用」に関しては,海底ケーブルの上陸点において光ケーブルの通信を複製して政府側に分岐させる装置の設置を検討している.通信の内容ではなく,IPアドレスなどの通信制業や宛先に関する情報を取得する.

「アクセス・無害化措置」では,サイバー攻撃の兆候を認めた場合にはサーバーに侵入しハッカーの行動を封じるようにする.

一方で恣意的な運用への歯止めもカギになる.政府は独立した第三者組織「サイバー通信情報監理委員会」で通信情報の取り扱いやアクセス無害化の承認を行う.

\subsection{感じたこと}

第三者組織の設置は重要であると思うが,無害化措置の対応スピードを確保することも重要である?


\end{document}