\documentclass{ltjsarticle}

\title{日経新聞夕刊}
\author{}
\date{\today}

\begin{document}
\maketitle

\section{ユン大統領を罷免}

韓国の憲法裁判所はユン大統領を罷免する決定を下した.ユン大統領は即時失職し,60日内に大統領選が行われる.


\subsection{背景}

ユン大統領は2024年12月,韓国野党が予算削減や政府高官の弾劾訴追を繰り返したことを理由に,「非常戒厳」宣言.戒厳司令部を設置し,言論統制や兵力を国会に侵入させるなどした.法的要件を満たさずに非常戒厳を宣言したとして,議員らは弾劾訴追案を決議し,ユン大統領は職務を停止されていた.

\subsection{詳細}

韓国大統領の弾劾裁判は国会の過半数で発議され,3分の2以上の賛成で可決される.過去に2例あり,2004年のノ・ムヒョン氏(不正資金疑惑,棄却され後に職務復帰)と2017年のパク・クネ氏(友人の国政介入疑惑,罷免が決定)があった.

大統領選では最大野党「共に民主党」のイ・ジェミョン氏が世論をリードしている.韓国総合株価指数(KOSPI)はトランプ関税の影響を受け続落したが,ユン大統領罷免による政治不安の解消から下げ幅は限定的であった.





\section{カナダ,アメリカに報復関税}

カナダのカーニー首相は米国から輸入する自動車に25\%の関税をかけることを発表した.\emph{アメリカ・メキシコ・カナダ協定(USMCA)}に準拠する自動車や部品は除外される.

\subsection{用語}

\begin{itemize}
  \item \emph{アメリカ・メキシコ・カナダ協定(USMCA)}とは,同3か国の間の自由貿易協定
\end{itemize}




\section{日経平均1200円超安}

トランプ関税の影響はいまだ続き,日経平均は一時34,000円を割った.景気悪化の懸念から,日銀の追加利上げも動きづらくなるとの見方が広がっている\footnote{また,リスク回避的な行動によって国債価格は上昇,すなわち利回りの低下が起きている.}.

米消費株も売りが拡大している.米消費企業の原料調達先であるアジア等にも大幅なトランプ関税がかけられているため,原価上昇は避けられない.



\section{実質消費支出,うるう年の影響を除けばプラス}

総務省が発表した2人以上世帯の消費支出は,昨年がうるう年だった影響を除き,物価変動の影響を除いた実質で,プラス1.8\%であった.総務省は「消費は回復傾向にある」と説明している.品目別にみると,食料品・住宅修繕・衣類はマイナスであるが,水道光熱が全体を押し上げる形となった.寒波の影響がみられている.また交通・通信もプラスであった.

一方,勤労者世帯の実収入は2.3\%減少であった.


\section{感じたこと}

\begin{itemize}
  \item 食料・衣料品の消費が抑えられている
  \item 寒波の影響で水道光熱支出が増えるのは消費が刺激されたとは言えないのでは.少なくとも持続的な消費支出ではない.
  \item 実収入が減少していることから,以前として賃金上昇率は十分な水準に達していないのでは
\end{itemize}追加利上げはより慎重に行うべき?




\section{マクラーレン・オートモーティブがイギリス電気自動車事業と経営統合}

イギリス高級車メーカーのマクラーレンが電気自動車新興企業のフォーセブンと経営統合すると発表.





\end{document}