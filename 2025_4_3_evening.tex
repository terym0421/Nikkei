\documentclass{ltjsarticle}

\title{日経新聞夕刊}
\author{}
\date{\today}

\begin{document}
\maketitle

\section{米相互関税 日本24\%}

トランプ大統領は世界各国からの輸入品に対して相互関税をかけると発表,多くの国に対いて一律10\%に加え,日本に対しては合計24\%の追加関税

\subsection{背景}

アメリカは現在
\begin{itemize}
  \item 年1.2兆円の貿易赤字
  \item 国内産業の\emph{空洞化}
\end{itemize}
を抱えている.

\subsection{詳細}

すべての国にかける相互関税は「基本税率」と「上乗せ税率」の2つからなり,上乗せ税率によって相手国との関税率を同水準に調整するねらい.自動車・鉄鋼・アルミニウム等は分野別で追加関税をかけるため,この上乗せ税率の対象からは除外される.上乗せ税率の対象国はおよそ60か国で日本も含まれる.

アメリカの主張の一部に「日本はコメに700\%の関税をかけている」としているが,日本政府は「\emph{輸入差益のみで輸入し,関税は無税としている}」と反論.

一方,日本自動車や部品には25\%の追加関税が発動.

$\rightarrow $長期的にみれば外国製品の流入を防ぎ,米国内産業の発展につながるが,短期的には消費者に負担を強いることとなる.


\subsection{用語・わからなかった点}

\begin{itemize}
  \item 産業の空洞化とは,国内生産していた製造品が,物価や人件費の安さ,為替等の問題によって海外に移転してしまう減少.これによって雇用機会の喪失,貿易赤字,技術力の低下による国際競争力の喪失などの問題が発生する恐れがある.
  \item 日本政府の反論について,日本はミニマムアクセス米といって,必要最低限のコメを海外から輸入している.貿易自由化の流れをくみ1986年ウルグアイ・ラウンド交渉で日本は米の輸入を行う義務が生じた.その際日本国内で生産される米を保護する目的で,無税枠(77万トンまで)と課税枠を設定.
  輸入したコメは食用の他,海外援助や飼料に使われている.
\end{itemize}


\subsection{感じたこと}

\begin{itemize}
  \item 自動車産業や輸出業界に打撃?株価低下の可能性
  \item 短期的にはアメリカの消費が抑えられるため景気後退,円高の可能性
\end{itemize}



\section{日経平均 一時1600円超安}

トランプ関税の企業業績への悪影響の懸念から日経平均は1600円以上低下し34,100円台をつけ,ダウ工業株30種平均も一時1000ドル以上低下.またドル円相場は一時147円台まで円高が進んだ.


\subsection{用語・わからなかった点}

\begin{itemize}
  \item ダウ工業株30種平均,いわゆるダウ平均は,S\&Pダウ・ジョーンズ・インデックス社が公表している.アメリカ国内の様々な業種の銘柄を30種選定し,その単純な平均値をとったもの.日経225の米版のようなもの.
\end{itemize}

\subsection{感じたこと}

株価低迷はどれくらい続くか,トランプ関税を受けた各社の反応や日本政府の関税撤廃に向けた動きが明らかになれば投資家の反応も落ち着くと考えられる?


\section{TikTok米事業売却先はファンド連合有力に}

TikTokのアメリカにおける事業の売却先としてアメリカ投資ファンド連合が有力視されていることが明らかになった.

\subsection{背景}

米政府はtiktokによってアメリカ顧客の大規模なデータ流出が行われていると主張し,tiktok米事業を売却して親会社から独立するか,米国内でのサービス提供を禁止する「tiktok規制法」を定めていた.最終的な判断はトランプにゆだねられていたが,株式の8割を米投資家に売却させ,独立させる案がまとまった.


\section{米小売業,消費悲観ムード/全米車労働組合は関税支持}

アメリカ小売業協会(NRF)はオンラインイベントで,参加した小売業の経営者らから,トランプ関税が与える消費への影響について意見交換を行った.多くの経営者は,トランプ関税は消費減衰を招くと予想しており,アメリカ小売業の成長率の見通しは2.7\%~3.7\%とコロナショック以来の低水準であるとした.加えて,貿易政策と関税のバランスの取れたアプローチを追求するべきと求めた.

一方全米自動車労働組合(UAW)は,トランプ関税を支持すると表明し,「自由貿易は雇用流出を招く災厄である」と述べた.UAWはゼネラルモーターズやフォード社など「ビッグ3」が主な会員.彼らの生産拠点は米北東部のラストベルトに集中しており,そこはかつては民主党の支持層だった製造業労働者が集まっており,産業の空洞化に伴って大統領選では共和党に流れた.


\subsection{感じたこと}

トランプ関税は支持者獲得の手段の一つであろう.はじめは過激で少し非現実的だが,有権者にとっては魅力的な政策を掲げ,実際には水面下で現実的な落としどころを探るつもり?今後各政府との交渉でトランプ関税は撤廃もしくは低水準に落ち着くのではないか.


\section{テスラ株,一転5\%上昇}

テスラ株は,イーロンマスクのトランプ政権離脱の報道を受け,5\%値上がりありした.

\subsection{背景}

米大統領選においてイーロンマスクはトランプ支持を表明し,また,トランプ大統領就任後は,政府効率化省を率いて連邦政府の人員削減を主導していた.そのため反トランプ派/連邦政府リストラへの反発によるテスラ不買運動が広がり,株価は低迷していた.



\section{10年物国債,表面利率1.4\%}

財務省は4月発行の10年もの国債の\emph{表面利率}を1.4\%に引き上げることを決定した.

\subsection{背景}

日銀の追加利上げの予想から,国内金利が上昇していることを受け,表面利率を見直した.表面利率が,市場に出回っている国債の利回りよりも低すぎる場合,落札価格が額面を大きく下回る可能性があるため,利率を見直す必要が生じた.


\subsection{用語・わからなかった点}

\begin{itemize}
  \item 表面利率とは,国債の発行時に設定する額面に対する利率のこと.国債売買時に計算される利回り(一般にいう金利)とは区別される.
\end{itemize}


\end{document}